%Chapter 0
\chapter{0}{Fundamentals}
\subsection*{0.1 Evaluating a Polynomial}
\begin{task}{0.1.1 a)}
	\[P(x)=6x^4+x^3+5x^2+x+1=1+x(1+x(5+x(1+6x)))\]
	With nested:
	\[6\*1/3+1=3\]
	\[3\*1/3+5=6\]
	\[6\*1/3+1=3\]
	\[3\*1/3+1=2\]
	Without nested:
	\[6\*(1/3)^4+(1/3)^3+5\*(1/3)^2+1/3+1\]
	\[6/81+1/27+5/9+1/3+1\]
	\[6/81+3/81+45/81+27/81+81/81\]
	\[162/81=2\]
\end{task}

\begin{task}{b)}
	\[P(x)=-3x^4+4x^3+5x^2-5x+1=1+x(-5+x(5+x(4-3x)))\]
	With nested:
	\[-3\*1/3+4=3\]
	\[3\*1/3+5=6\]
	\[6\*1/3-5=-3\]
	\[-3\*1/3+1=0\]
	Without nested:
	\[-3\*(1/3)^4+4\*(1/3)^3+5\*(1/3)^2-5\*(1/3)+1\]
	\[-3/81+4/27+5/9-5/3+1\]
	\[-3/81+12/81+45/81-135/81+81/81\]
	\[0/81=0\]
\end{task}

\begin{task}{c)}
\[P(x)=2x^4+x^3-x^2+1=1+x(0+x(-1+x(1+2x)))\]
With nested:
\[2\*1/3+1=5/3\]
\[5/3\*1/3-1=-4/9\]
\[-4/9\*1/3=-4/27\]
\[-4/27\*1/3+1=77/81\]
Without nested:
\[2\*(1/3)^4+(1/3)^3-(1/3)^2+1\]
\[2/81+1/27-1/9+1\]
\[2/81+3/81-9/81+81/81\]
\[77/81\]
\end{task}

\begin{task}{0.1.2 a)}
	\[P(x)=6x^3-2x^2-3x+7=7+x(-3+x(-2+6x))\]
	With nested:
	\[6\*(-1/2)-2=-5\]
	\[-5\*(-1/2)-3=-1/2\]
	\[-1/2\*(-1/2)+7=29/4\]
\end{task}

\begin{task}{b)}
\[P(x)=8x^5-x^4-3x^3+x^2-3x+1=1+x(-3+x(1+x(-3+x(-1+8x))))\]
With nested:
\[8\*(-1/2)-1=-5\]
\[-5\*(-1/2)-3=-1/2\]
\[-1/2\*(-1/2)+1=5/4\]
\[5/4\*(-1/2)-3=-29/8\]
\[-29/8\*(-1/2)+1=45/16\]
\end{task}

\begin{task}{c)}
\[P(x)=4x^6-2x^4-2x+4=4+x(-2+x(0+x(0+x(-2+x(0+4x)))))\]
With nested:
\[4\*(-1/2)=-2\]
\[-2\*(-1/2)-2=-1\]
\[-1\*(-1/2)=1/2\]
\[1/2\*(-1/2)=-1/4\]
\[-1/4\*(-1/2)-2=-15/8\]
\[-15/8\*(-1/2)+4=79/16\]
\end{task}

\begin{task}{0.1.3}
\[P(x)=x^6-4x^4+2x^2+1=1+x^2(2+x^2(-4+x^2))\]
With nested:
\[(1/2)^2-4=-15/4\]
\[-15/4\*(1/2)^2+2=17/16\]
\[17/16\*(1/2)^2+1=81/64\]
\end{task}

\begin{task}{0.1.4 a)}
\[P(x)=1+x(1/2+(x-2)(1/2+(x-3)(-1/2)))\]
With nested:
\[-1/2\*(5-3)+1/2=-1/2\]
\[-1/2\*(5-2)+1/2=-1\]
\[-1\*5+1=-4\]
\end{task}

\begin{task}{b)}
\[P(x)=1+x(1/2+(x-2)(1/2+(x-3)(-1/2)))\]
With nested:
\[-1/2\*(-1-3)+1/2=5/2\]
\[5/2\*(-1-2)+1/2=-7\]
\[-7\*(-1)+1=8\]
\end{task}

\begin{task}{0.1.5 a)}
\[P(x)=4+x(4+(x-1)(1+(x-2)(3+2(x-3))))\]
With nested:
\[2\*(1/2-3)+3=-2\]
\[-2\*(1/2-2)+1=4\]
\[4\*(1/2-1)+4=2\]
\[2\*(1/2)+4=5\]
\end{task}

\begin{task}{b)}
\[P(x)=4+x(4+(x-1)(1+(x-2)(3+2(x-3))))\]
With nested:
\[2\*(-1/2-3)+3=-4\]
\[-4\*(-1/2-2)+1=11\]
\[11\*(-1/2-1)+4=-25/2\]
\[-25/2\*(-1/2)+4=41/4\]
\end{task}

\begin{task}{0.1.6 a)}
	\[P(x)=
	a_0+a_5x^5+a_{10}x^{10}+a_{15}x^{15}=
	a_0+x^5(a_{5}+x^{5}(a_{10}+x^{5}(a_{15})))\]
	\[a_{15}x^{5}+a_{10}=b_1\qquad\text{5 multiplications and 1 addition}\]
	\[b_1x^{5}+a_{5}=b_2\qquad\text{(since $x^5$ is calculated) 1 multiplications and 1 addition}\]
	\[b_2x^{5}+a_0=b_3\qquad\text{1 multiplications and 1 addition}\]
	$5+1+1=7$ multiplications, $1+1+1=3$ addition.
\end{task}

\begin{task}{b)}
\[P(x)=
a_7x^7+a_{12}x^{12}+a_{17}x^{17}+a_{22}x^{22}+a_{27}x^{27}=
x^7(a_{7}+x^{5}(a_{12}+x^{5}(a_{17}+x^5(a_{22}+x^{5}(a_{27})))))\]
\[a_{27}x^{5}+a_{22}=b_1\qquad\text{5 multiplications and 1 addition}\]
\[b_1x^{5}+a_{17}=b_2\qquad\text{(since $x^5$ is calculated) 1 multiplications and 1 addition}\]
\[b_2x^{5}+a_{12}=b_3\qquad\text{1 multiplications and 1 addition}\]
\[b_3x^{5}+a_{7}=b_4\qquad\text{1 multiplications and 1 addition}\]
\[b_4x^{7}=b_4\qquad\text{2 multiplications}\]
$5+1+1+1+2=10$ multiplications, $1+1+1+1=4$ addition.
\end{task}

\begin{task}{0.1.7}
$n$ multiplications, $2n$ addition.
\end{task}

\begin{task}{(c) 0.1.1}
\lstinputlisting[language=Matlab]{p011.m}
Output:

\texttt{p=51.012752082749991}

\texttt{q=51.012752082745230}

\texttt{estError=0.000000000004761}

\end{task}

\begin{task}{(c) 0.1.2}
\begin{align*}
P(x)=&
1-x+x^2-x^3+\ldots+x^{98}-x^{99}=
1-x+x^2(1-x)+\ldots+x^{98}(1-x)= \\
&\sum_{k=0}^{49}x^{2k}(1-x)=
(1-x)\sum_{k=0}^{49}(x^2)^k=
(1-x)\dfrac{1-(x^2)^{50}}{1-x}=
1-x^{100}
\end{align*}
\lstinputlisting[language=Matlab]{p012.m}
Output:

\texttt{p=-0.000500245079648}

\texttt{q=-0.001000495161746}

\texttt{estError=0.000500250082098}

\end{task}

\subsection*{0.2 Binary Numbers}

\begin{task}{0.2.1 a)}
\[64/2=32\text{ R }0\]
\[32/2=16\text{ R }0\]
\[16/2=8\text{ R }0\]
\[8/2=4\text{ R }0\]
\[4/2=2\text{ R }0\]
\[2/2=1\text{ R }0\]
\[1/2=0\text{ R }1\]
\[(64)_{10}=(1000000)_2\]
\end{task}

\begin{task}{b)}
\[17/2=8\text{ R }1\]
\[8/2=4\text{ R }0\]
\[4/2=2\text{ R }0\]
\[2/2=1\text{ R }0\]
\[1/2=0\text{ R }1\]
\[(17)_{10}=(10001)_2\]
\end{task}

\begin{task}{c)}
\[79/2=32\text{ R }1\]
\[39/2=19\text{ R }1\]
\[19/2=9\text{ R }1\]
\[9/2=4\text{ R }1\]
\[4/2=2\text{ R }0\]
\[2/2=1\text{ R }0\]
\[1/2=0\text{ R }1\]
\[(79)_{10}=(1001111)_2\]
\end{task}

\begin{task}{d)}
\[227/2=113\text{ R }1\]
\[113/2=56\text{ R }1\]
\[56/2=28\text{ R }0\]
\[28/2=14\text{ R }0\]
\[14/2=7\text{ R }0\]
\[7/2=3\text{ R }1\]
\[3/2=1\text{ R }1\]
\[1/2=0\text{ R }1\]
\[(227)_{10}=(11100011)_2\]
\end{task}

\begin{task}{0.2.2 a)}
\[1/8\*2=1/4\text{ R }0\]
\[1/4\*2=1/2\text{ R }0\]
\[1/2\*2=0\text{ R }1\]
\[(1/8)_{10}=(.001)_2\]
\end{task}

\begin{task}{b)}
\[7/8\*2=3/4\text{ R }1\]
\[3/4\*2=1/2\text{ R }1\]
\[1/2\*2=0\text{ R }1\]
\[(7/8)_{10}=(.111)_2\]
\end{task}

\begin{task}{c)}
It's larger than 2, factor it out.

Integer part:
\[2/2=1\text{ R }0\]
\[1/2=56\text{ R }1\]
Fractional part:
\[3/16\*2=3/8\text{ R }0\]
\[3/8\*2=3/4\text{ R }0\]
\[3/4\*2=1/2\text{ R }1\]
\[1/2\*2=0\text{ R }1\]
\[(35/16)_{10}=(10.0011)_2\]
\end{task}

\begin{task}{d)}
\[31/64\*2=31/32\text{ R }0\]
\[31/32\*2=15/16\text{ R }1\]
\[15/16\*2=7/8\text{ R }1\]
\[7/8\*2=3/4\text{ R }1\]
\[3/4\*2=1/2\text{ R }1\]
\[1/2\*2=0\text{ R }1\]
\[(31/64)_{10}=(.011111)_2\]
\end{task}

\begin{task}{0.2.3 a)}
Solve the integer and fractional part separately.
	
Integer part:
\[10/2=5\text{ R }0\]
\[5/2=2\text{ R }1\]
\[2/2=1\text{ R }0\]
\[1/2=0\text{ R }1\]
Fractional part:
\[.5\*2=0\text{ R }1\]
Sum:
\[(10.5)_{10}=(1010.1)_2\]
\end{task}

\begin{task}{b)}
\[1/3\*2=2/3\text{ R }0\]
\[2/3\*2=1/3\text{ R }1\]
\[1/3\*2=2/3\text{ R }0\]
The period is two.
\[(1/3)_{10}=(.\overline{01})_2\]
\end{task}

\begin{task}{c)}
\[5/7\*2=3/7\text{ R }1\]
\[3/7\*2=6/7\text{ R }0\]
\[6/7\*2=5/7\text{ R }1\]
\[5/7\*2=3/7\text{ R }1\]
The period is three.
\[(5/7)_{10}=(.\overline{101})_2\]
\end{task}

\begin{task}{d)}
Solve the integer and fractional part separately.

Integer part:
\[12/2=6\text{ R }0\]
\[6/2=3\text{ R }0\]
\[3/2=1\text{ R }1\]
\[1/2=0\text{ R }1\]
Fractional part:
\[.8\*2=.6\text{ R }1\]
\[.6\*2=.2\text{ R }1\]
\[.2\*2=.4\text{ R }0\]
\[.4\*2=.8\text{ R }0\]
\[.8\*2=.6\text{ R }1\]
The period is four.

Sum:
\[(12.8)_{10}=(1100.\overline{1100})_2\]
\end{task}

\begin{task}{e)}
Solve the integer and fractional part separately.

Integer part:
\[55/2=27\text{ R }1\]
\[27/2=13\text{ R }1\]
\[13/2=6\text{ R }1\]
\[6/2=3\text{ R }0\]
\[3/2=1\text{ R }1\]
\[1/2=0\text{ R }1\]
Fractional part:
\[.4\*2=.8\text{ R }0\]
\[.8\*2=.6\text{ R }1\]
\[.6\*2=.2\text{ R }1\]
\[.2\*2=.4\text{ R }0\]
\[.4\*2=.8\text{ R }0\]
The period is four.

Sum:
\[(55.4)_{10}=(110111.\overline{0110})_2\]
\end{task}

\begin{task}{f)}
\[.1\*2=.2\text{ R }0\]
\[.2\*2=.4\text{ R }0\]
\[.4\*2=.8\text{ R }0\]
\[.8\*2=.6\text{ R }1\]
\[.6\*2=.2\text{ R }1\]
\[.2\*2=.4\text{ R }0\]
The period is four after first bit.
\[(0.1)_{10}=(0.0\overline{0011})_2\]
\end{task}

\begin{task}{0.2.4 a)}
Solve the integer and fractional part separately.

Integer part:
\[11/2=5\text{ R }1\]
\[5/2=2\text{ R }1\]
\[2/2=1\text{ R }0\]
\[1/2=0\text{ R }1\]
Fractional part:
\[.25\*2=.5\text{ R }0\]
\[.5\*2=0\text{ R }1\]
Sum:
\[(11.25)_{10}=(1101.01)_2\]
\end{task}

\begin{task}{b)}
\[2/3\*2=1/3\text{ R }1\]
\[1/3\*2=2/3\text{ R }0\]
\[2/3\*2=1/3\text{ R }1\]
The period is two.
\[(2/3)_{10}=(.\overline{10})_2\]
\end{task}

\begin{task}{c)}
\[3/5 = 0.6\]
\[.6\*2=.2\text{ R }1\]
\[.2\*2=.4\text{ R }0\]
\[.4\*2=.8\text{ R }0\]
\[.8\*2=.6\text{ R }1\]
\[.6\*2=.2\text{ R }1\]
The period is four.
\[(3/5)_{10}=(.\overline{1001})_2\]
\end{task}

\begin{task}{d)}
Solve the integer and fractional part separately.

Integer part:
\[3/2=1\text{ R }1\]
\[1/2=0\text{ R }1\]
Fractional part:
\[.2\*2=.4\text{ R }0\]
\[.4\*2=.8\text{ R }0\]
\[.8\*2=.6\text{ R }1\]
\[.6\*2=.2\text{ R }1\]
\[.2\*2=.4\text{ R }0\]
The period is four.

Sum:
\[(3.2)_{10}=(11.\overline{0011})_2\]
\end{task}

\begin{task}{e)}
Solve the integer and fractional part separately.

Integer part:
\[30/2=15\text{ R }0\]
\[15/2=7\text{ R }1\]
\[7/2=3\text{ R }1\]
\[3/2=1\text{ R }1\]
\[1/2=0\text{ R }1\]
Fractional part:
\[.6\*2=.2\text{ R }1\]
\[.2\*2=.4\text{ R }0\]
\[.4\*2=.8\text{ R }0\]
\[.8\*2=.6\text{ R }1\]
\[.6\*2=.2\text{ R }1\]
The period is four.

Sum:
\[(30.6)_{10}=(11110.\overline{1001})_2\]
\end{task}

\begin{task}{f)}
Solve the integer and fractional part separately.

Integer part:
\[99/2=49\text{ R }1\]
\[49/2=24\text{ R }1\]
\[24/2=12\text{ R }0\]
\[12/2=6\text{ R }0\]
\[6/2=3\text{ R }0\]
\[3/2=1\text{ R }1\]
\[1/2=0\text{ R }1\]
Fractional part:
\[.9\*2=.8\text{ R }1\]
\[.8\*2=.6\text{ R }1\]
\[.6\*2=.2\text{ R }1\]
\[.2\*2=.4\text{ R }0\]
\[.4\*2=.8\text{ R }0\]
\[.8\*2=.6\text{ R }1\]
The period is four after the first bit.

Sum:
\[(99.9)_{10}=(1100011.1\overline{1100})_2\]
\end{task}

\begin{task}{0.2.5}
Solve the integer and fractional part separately. At least 4 decimal points ($3.1416$) will give the correct answer.

Integer part:
\[3/2=1\text{ R }1\]
\[1/2=0\text{ R }1\]
Fractional part:
\[.14159265358979\*2=.28318530717958\text{ R }0\]
\[.28318530717958\*2=.56637061435916\text{ R }0\]
\[.56637061435916\*2=.13274122871832\text{ R }1\]
\[.13274122871832\*2=.26548245743664\text{ R }0\]
\[.26548245743664\*2=.53096491487328\text{ R }0\]
\[.53096491487328\*2=.06192982974656\text{ R }1\]
\[.06192982974656\*2=.12385965949312\text{ R }0\]
\[.12385965949312\*2=.24771931898624\text{ R }0\]
\[.24771931898624\*2=.49543863797248\text{ R }0\]
\[.49543863797248\*2=.99087727594496\text{ R }0\]
\[.99087727594496\*2=.98175455188992\text{ R }1\]
\[.98175455188992\*2=.96350910377984\text{ R }1\]
\[.96350910377984\*2=.92701820755968\text{ R }1\]
Sum:
\[(\pi)_{10}\approx(11.0010010000111)_2\]
\end{task}

\begin{task}{0.2.6}
Do it the same way as in the last exercise. At least 4 decimal points ($2.7183$) will give the correct answer.
\[(e)_{10}\approx(10.1011011111100)_2\]
\end{task}

\begin{task}{0.2.7 a)}
\[2^6+2^4+2^2+1=64+16+4+1=85\]
\end{task}

\begin{task}{b)}
Solve the integer and fractional part separately.

Integer part:
\[2^3+2^1+1=8+2+1=11\]
Fractional part:
\[1/2+1/8=5/8=.625\]
Sum:
\[(1011.101)_2=(11.625)_{10}\]
\end{task}

\begin{task}{c)}
Solve the integer and fractional part separately.

Integer part:
\[2^4+2^2+2^1+1=16+4+2+1=23\]
Fractional part:
\[x=(.\overline{01})_2\]
\[2^2x=(01.\overline{01})_2\]
\[(2^2-1)x=(01.\overline{01})_2-(.\overline{01})_2=(01)_2=1\lra 
x= \dfrac{1}{4-1}=1/3\]
Sum:
\[(10111.\overline{01})_2=(23+1/3)_{10}=(70/3)_{10}\]
\end{task}

\begin{task}{d)}
Solve the integer and fractional part separately.

Integer part:
\[2^2+2^1=4+2=6\]
Fractional part:
\[x=(.\overline{10})_2\]
\[2^2x=(10.\overline{10})_2\]
\[(2^2-1)x=(10.\overline{10})_2-(.\overline{10})_2=(10)_2=2\lra 
x= \dfrac{2}{4-1}=2/3\]
Sum:
\[(110.\overline{10})_2=(6+2/3)_{10}=(20/3)_{10}\]
\end{task}

\begin{task}{e)}
Solve the integer and fractional part separately.

Integer part:
\[2^1=2\]
Fractional part:
\[x=(.\overline{110})_2\]
\[2^3x=(110.\overline{110})_2\]
\[(2^3-1)x=(110.\overline{110})_2-(.\overline{110})_2=(110)_2=6\lra 
x= \dfrac{6}{8-1}=6/7\]
Sum:
\[(10.\overline{110})_2=(2+6/7)_{10}=(20/7)_{10}\]
\end{task}

\begin{task}{f)}
Solve the integer and fractional part separately.

Integer part:
\[2^2+2^1=4+2=6\]
Fractional part:
\[x=(.1\overline{101})_2\]
\[y=2x=(1.\overline{101})_2\]
\[z=(.\overline{101})_2\]
\[2^3z=(101.\overline{101})_2\]
\[(2^3-1)z=(101.\overline{101})_2-(.\overline{101})_2=(101)_2=5\lra 
z= \dfrac{5}{8-1}=5/7\]
\[y=1+5/7=12/7\lra x = y/2 = 6/7\]
Sum:
\[(110.1\overline{101})_2=(6+6/7)_{10}=(48/7)_{10}\]
\end{task}

\begin{task}{g)}
Solve the integer and fractional part separately.

Integer part:
\[2^1=2\]
Fractional part:
\[x=(.010\overline{1101})_2\]
\[y=2^3x=(010.\overline{1101})_2\]
\[z=(.\overline{1101})_2\]
\[2^4z=(1101.\overline{1101})_2\]
\[(2^4-1)z=(1101.\overline{1101})_2-(.\overline{1101})_2=(1101)_2=13\lra 
z= \dfrac{13}{16-1}=13/15\]
\[y=2+13/15=43/15\lra x = y/8 = 43/120\]
Sum:
\[(10.010\overline{1101})_2=(2+43/120)_{10}=(283/120)_{10}\]
\end{task}

\begin{task}{h)}
Solve the integer and fractional part separately.

Integer part:
\[2^2+2^1+1=4+2+1=7\]
Fractional part:
\[x=(.\overline{1})_2\]
\[2x=(1.\overline{1})_2\]
\[(2-1)x=(1.\overline{1})_2-(.\overline{1})_2=(1)_2=1\lra 
x= \dfrac{1}{2-1}=1\]
Sum:
\[(111.\overline{1})_2=(7+1)_{10}=(8)_{10}\]
\end{task}

\begin{task}{0.2.8 a)}
\[2^4+2^3+2^1+1=16+8+2+1=27\]
\end{task}

\begin{task}{b)}
Solve the integer and fractional part separately.

Integer part:
\[2^5+2^4+2^2+2^1+1=32+16+4+2+1=55\]
Fractional part:
\[1/8=.125\]
Sum:
\[(110111.001)_2=(55.125)_{10}\]
\end{task}

\begin{task}{c)}
Solve the integer and fractional part separately.

Integer part:
\[2^2+2^1+1=4+2+1=7\]
Fractional part:
\[x=(.\overline{001})_2\]
\[2^3x=(001.\overline{001})_2\]
\[(2^3-1)x=(001.\overline{001})_2-(.\overline{001})_2=(001)_2=1\lra 
x= \dfrac{1}{8-1}=1/7\]
Sum:
\[(111.\overline{001})_2=(7+1/7)_{10}=(50/7)_{10}\]
\end{task}

\begin{task}{d)}
Solve the integer and fractional part separately.

Integer part:
\[2^3+2^1=8+2=10\]
Fractional part:
\[x=(.\overline{01})_2\]
\[2^2x=(01.\overline{01})_2\]
\[(2^2-1)x=(01.\overline{01})_2-(.\overline{01})_2=(01)_2=1\lra 
x= \dfrac{1}{4-1}=1/3\]
Sum:
\[(1010.\overline{01})_2=(10+1/3)_{10}=(31/3)_{10}\]
\end{task}

\begin{task}{e)}
Solve the integer and fractional part separately.

Integer part:
\[2^4+2^2+2^1+1=16+4+2+1=23\]
Fractional part:
\[x=(.1\overline{0101})_2\]
\[y=2x=(1.\overline{0101})_2\]
\[z=(.\overline{0101})_2\]
\[2^4z=(0101.\overline{0101})_2\]
\[(2^4-1)z=(0101.\overline{0101})_2-(.\overline{0101})_2=(0101)_2=5\lra 
z= \dfrac{5}{16-1}=1/3\]
\[y=1+1/3=4/3\lra x = y/2 = 2/3\]
Sum:
\[(10111.1\overline{0101})_2=(23+2/3)_{10}=(71/3)_{10}\]
\end{task}

\begin{task}{f)}
Solve the integer and fractional part separately.

Integer part:
\[2^3+2^2+2^1+1=8+4+2+1=15\]
Fractional part:
\[x=(.010\overline{001})_2\]
\[y=2^3x=(010.\overline{001})_2\]
\[z=(.\overline{001})_2\]
\[2^3z=(001.\overline{001})_2\]
\[(2^3-1)z=(001.\overline{001})_2-(.\overline{001})_2=(001)_2=1\lra 
z= \dfrac{1}{8-1}=1/7\]
\[y=2+1/7=15/7\lra x = y/8 = 15/56\]
Sum:
\[(1111.010\overline{001})_2=(15+15/56)_{10}=(855/56)_{10}\]
\end{task}

\subsection*{0.3 Floating Point Representation of Real Number}

\begin{task}{0.3.1 a)}
Covert decimal to binary.
\[1/4\*2=1/2\text{ R }0\]
\[1/2\*2=0\text{ R }1\]
\[(1/4)_{10}=(.01)_2\]
Left-justify it by shifting it twice.
\[(1/4)_{10}=1.000\ldots000 \times 2^{-2}\]
\end{task}

\begin{task}{b)}
Covert decimal to binary.
\[1/3\*2=2/3\text{ R }0\]
\[2/3\*2=1/3\text{ R }1\]
\[1/3\*2=2/3\text{ R }0\]
\[(1/3)_{10}=(.\overline{01})_2\]
Left-justify it by shifting it twice. Since the 53 bit will be a zero, round down (do nothing).
\[(1/3)_{10}=1.0101\ldots01 \times 2^{-2}\]
\end{task}

\begin{task}{c)}
Covert decimal to binary.
\[2/3\*2=1/3\text{ R }1\]
\[1/3\*2=2/3\text{ R }0\]
\[2/3\*2=1/3\text{ R }1\]
\[(2/3)_{10}=(.\overline{10})_2\]
Left-justify it by shifting it once. Since the 53 bit will be a zero, round down (do nothing).
\[(1/3)_{10}=1.0101\ldots01 \times 2^{-1}\]
\end{task}

\begin{task}{d)}
Covert decimal to binary.
\[0.9\*2=0.8\text{ R }1\]
\[0.8\*2=0.6\text{ R }1\]
\[0.6\*2=0.2\text{ R }1\]
\[0.2\*2=0.4\text{ R }0\]
\[0.4\*2=0.8\text{ R }0\]
\[0.8\*2=0.6\text{ R }1\]
\[(0.9)_{10}=(.1\overline{1100})_2\]
Left-justify it by shifting it once. Since the 53 bit will be a one and has following non-zero bits, round up.
\[(1/3)_{10}=1.1100\ldots1101 \times 2^{-1}\]
\end{task}

\begin{task}{0.3.2 a)}
Covert decimal to binary. Solve the integer and fractional part separately.

Integer part:
\[9/2=4\text{ R }1\]
\[4/2=2\text{ R }0\]
\[2/2=1\text{ R }0\]
\[1/2=0\text{ R }1\]
Fractional part:
\[.5\*2=0\text{ R }1\]
Sum:
\[(9.5)_{10}=(1001.1)_2\]
Left-justify it by shifting it three times then pad with zeros.
\[(9.5)_{10}=1.00110\ldots00 \times 2^{3}\]
\end{task}

\begin{task}{b)}
Covert decimal to binary. Solve the integer and fractional part separately.

Integer part:
\[9/2=4\text{ R }1\]
\[4/2=2\text{ R }0\]
\[2/2=1\text{ R }0\]
\[1/2=0\text{ R }1\]
Fractional part:
\[.6\*2=.2\text{ R }1\]
\[.2\*2=.4\text{ R }0\]
\[.4\*2=.8\text{ R }0\]
\[.8\*2=.6\text{ R }1\]
\[.6\*2=.2\text{ R }1\]
Sum:
\[(9.6)_{10}=(1001.\overline{1001})_2\]
Left-justify it by shifting it three times. Since the 53 bit will be a zero, round down (do nothing).
\[(9.6)_{10}=1.0011\ldots0011 \times 2^{3}\]
\end{task}

\begin{task}{c)}
Covert decimal to binary. Solve the integer and fractional part separately.

Integer part:
\[100/2=50\text{ R }0\]
\[50/2=25\text{ R }0\]
\[25/2=12\text{ R }1\]
\[12/2=6\text{ R }0\]
\[6/2=3\text{ R }0\]
\[3/2=1\text{ R }1\]
\[1/2=0\text{ R }1\]
Fractional part:
\[.2\*2=.4\text{ R }0\]
\[.4\*2=.8\text{ R }0\]
\[.8\*2=.6\text{ R }1\]
\[.6\*2=.2\text{ R }1\]
\[.2\*2=.4\text{ R }0\]
Sum:
\[(100.2)_{10}=(1100100.\overline{0011})_2\]
Left-justify it by shifting it 6 times. Since the 53 bit will be a one and has following non-zero bits, round up.
\[(100.2)_{10}=1.1001000011001100\ldots11001101 \times 2^{6}\]
\end{task}

\begin{task}{d)}
Covert decimal to binary. Solve the integer and fractional part separately.

Integer part:
\[6/2=3\text{ R }0\]
\[3/2=1\text{ R }1\]
\[1/2=0\text{ R }1\]
Fractional part:
\[2/7\*2=4/7\text{ R }0\]
\[4/7\*2=1/7\text{ R }1\]
\[1/7\*2=2/7\text{ R }0\]
\[2/7\*2=4/7\text{ R }0\]
Sum:
\[(44/7)_{10}=(110.\overline{010})_2\]
Left-justify it by shifting it twice. Since the 53 bit will be a zero, round down (do nothing).
\[(44/7)_{10}=1.100100100\ldots001001 \times 2^{2}\]
\end{task}

\begin{task}{0.3.3}
Since $(5)_{10}=(101)_2$ and $(2^{-k})_{10}=(0.00\ldots001)_2$ where the number of zeros is equal to $k$, the sum will be $101.\underbrace{00\ldots00}_{k - 1 \text{ zeros}}1$. In the IEEE format the right-most 1 will be at the $(k+2)$th bit. For the number to be represented exactly in double precision $k+2 \le 52 \lra k \le 50$. Since $k$ is a positive integer, $1 \le k \le 50$.
\end{task}

\begin{task}{0.3.4}
Since $(19)_{10}=(10011)_2$ and $(2^{-k})_{10}=(0.00\ldots001)_2$ where the number of zeros is equal to $k$ the sum will be $10011.\underbrace{00\ldots00}_{k - 1 \text{ zeros}}1$. In the IEEE format the right-most 1 will be at the $(k+4)$th bit. If the 1 is at a position further away than the 52 bit it will be rounded down. This means if $k+4 > 52 \lra k > 48$ then $\text{fl}(19+2^{-k})=\text{fl}(19)$. The largest possible value of $k$ therefore is 48.
\end{task}